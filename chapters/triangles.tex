\documentclass[../main.tex]{subfile}
\begin{document}
\chapter{Triangles}
For a triangle defined with three vertices $A(x_1,y_1),B(x_2,y_2),C(x_3,y_3)$ and corresponding sides of length $a,b,c$, then:

\section{Centroid of a Traiangle}
	\begin{align}
			\text{Centroid of }\triangle ABC=(\dfrac{x_1+x_2+x_3}{3},\dfrac{y_1+y_2+y_3}{3})\\
	\end{align}

\section{Area of Triangle}

\subsection{Determinant Method}
	\begin{align}
		\text{Area of }\triangle ABC =\dfrac{1}{2}\begin{vmatrix}x_1&y_1&1\\x_2&y_2&1\\x_3&y_3&1\end{vmatrix}
	\end{align}

\subsection{Heron's Formula}
	For a triangle, the semi-perimeter, $s$, is defined as:
		\begin{align}
			s=\dfrac{a+b+c}{2}\nonumber
		\end{align}
	The area of the triangle can be defined as:
	\begin{align}
		\text{Area of }\triangle ABC = \sqrt{s \cdot (s-a) \cdot (s-b) \cdot (s-c)}
	\end{align}

\section{Incircle of a Triangle}
	The radius, $r$, and centre of incircle, $o$, is:
	\begin{align}
		o & = \left( \dfrac{ a x_1 + b x_2 + c x_3}{a + b + c},\dfrac{a y_1 + b y_2 + c y_3}{a + b + c} \right)\\
		r & = \sqrt{ \dfrac{ (s-a) \cdot (s-b) \cdot (s-c) }{s}}\\
	\end{align}

\section{Circumcircle of a Triangle}
	The radius, $R$, and centre, $O$, of circumcircle is defined as:
	\begin{align}
		O & = \left( \dfrac{x_1 \sin 2A + x_2 \sin 2B + x_3 \sin 2C}{\sin 2A + \sin 2B + \sin 2C}, \dfrac{y_1 \sin 2A + y_2 \sin 2B + y_3 \sin 2C}{\sin 2A + \sin 2B + \sin 2C} \right)\\
		2R & = \dfrac{a}{\sin A} = \dfrac{b}{\sin B} = \dfrac{c}{\sin C}
	\end{align}
\end{document}