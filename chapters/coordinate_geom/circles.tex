\documentclass[../main.tex]{subfile}

\begin{document}
    \chapter{Circles}
        \section{Locus Form}
        \begin{align}
            (x-g)^2+(y-h)^2=r^2
        \end{align}
        where the centre is $(g,h)$ and the radius is $r$.

        \section{Diameter Form}
        \begin{align}
            (x-a)(x-c)+(y-b)(y-d)=0
        \end{align}
        where $(a,b)$ and $(c,d)$ are the two ends of the diamter.

        \section{General Form}
        If the equation of a circle is in the form:
        \begin{align}
            x^2+y^2+2gx+2fy+c=0
        \end{align}
        Then the following is true about the circle:
        \begin{enumerate}
        \item centre of the circle is $(-g,-f)$
        \item radius of circle is $\sqrt{g^2+f^2-c}$
        \end{enumerate}

        \section{Important Relations}
        \begin{enumerate}
        \item If the circle passes through the origin, $g=0,f=0$.
        \item If the circle touches the x-axis $c=g^2$.
        \item If the circle touches the y-axis $c=f^2$.
        \end{enumerate}

        \section{Common for Two Circles}
        \begin{enumerate}
        \item The common chord passing between two circles $S_1$ and $S_2$ are:
        \begin{align}
        S_1-S_2=0
        \end{align}
        \item Circles passing through the intersection of two circles is:
        \begin{align}
        S_2+k(S_1-S_2)=0\text{ }\forall k\in\mathbb{R}
        \end{align}
        \end{enumerate}
\end{document}