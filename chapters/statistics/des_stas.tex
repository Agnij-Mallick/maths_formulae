\documentclass[../main.tex]{subfile}

\begin{document}
    \chapter{Descriptive Statistics}
    \section{Measure of Location}
    \subsection{Mean}
    \begin{align}
        \bar{x} = \dfrac{1}{n} \displaystyle\sum_{i=1}^{n} x_i
    \end{align}

    \subsection{Median}
    For odd number of elements in a dataset:
    \begin{align}
        \tilde{x} = x_{\frac{n+1}{2}}
    \end{align}
    For even number of elements in a dataset:
    \begin{align}
        \tilde{x} = \dfrac{x_{\frac{n}{2}}+x_{\left(\frac{n}{2}+1\right)}}{2}
    \end{align}

    \subsection{Mode}
    \begin{align}
        Mo(x) = \max(f(x_i))
    \end{align}

    \subsection{Quartile}
    Measure of percentage of elements less than or equal to a term

    \section{Measure of Spread}
    \subsection{Variance}
    Variance measured on the whole population
    \begin{align}
        \sigma^2 = \dfrac{1}{n} \sum_{i=1}^{n} \left( x_i - \bar{x} \right)^2
    \end{align}

    \subsection{Sample Variance}
    Variance measured on a sample population
    \begin{align}
        s^2 = \dfrac{1}{n-1} \sum_{i=1}^{n} \left( x_i - \bar{x} \right)^2
    \end{align}

    \subsection{Standard Deviation and Sample Standard}
    \begin{align}
        \sigma = \sqrt{\sigma^2}\\
        s=\sqrt{s^2}
    \end{align}

    \subsection{Co-efficient of Variance}
    \begin{align}
        v = \dfrac{s}{\bar{x}}
    \end{align}

    \section{Skewness}
    \subsection{Types of Skewness}
    \begin{table}[!h]
        \centering
        \begin{tabular}{c|c|l}
            Name & Other Name & Characteristic\\
            \hline
            Right Skew & Positive Skew & Data concentrated on the lower side\\
            Symmetric Distribution & Normal Distribution & Data distributed evenly\\
            Left Skew & Negative Skew & Data concentrated on the higher side
        \end{tabular}
    \end{table}

    \subsection{Measure of Skewness}
    Skewness is measured by the Moment Co-efficient of Skewness.
    \begin{align}
        g_m &= \dfrac{m_3}{s^3}, \text{ where}\\
        m_3 &= \dfrac{1}{n}\sum_{i=1}^{n} \left( x_i - \bar{x} \right)^3
    \end{align}

    \subsubsection{Type of Skewness}
    The type of skewness from the value is $g_m$ is:
    \begin{table}[h!]
        \centering
        \begin{tabular}{c|c}
            Value of $g_m$ & Type\\
            \hline
            $g_m = 0$ & Symmetric\\
            $g_m > 0$ & Positive Skew\\
            $g_m < 0$ & Negative Skew
        \end{tabular}
    \end{table}

    \subsubsection{Degree of Skewness}
    The degree of skewness from the value is $g_m$ is:
    \begin{table}[h!]
        \centering
        \begin{tabular}{c|c}
            Value of $g_m$ & Degree\\
            \hline
            $|g_m| > 1$ & High Skewness\\
            $0.5 <|g_m| \geq 1$ & Moderate Skewness\\
            $|g_m| \leq 0.5$ & Low Skewness
        \end{tabular}
    \end{table}

    \section{Kurtosis}
    Kurtosis is the measure of peakedness of data. Fisher's kurtosis measure is defined as:
    \begin{align}
        \gamma &= \dfrac{m_4}{s^4}, \text{ where}\\
        m_4 &= \dfrac{1}{n}\sum_{i=1}^{n} \left( x_i - \bar{x} \right)^4
    \end{align}

    \subsection{Type of Kurtosis}
    The types of kurtosis from the value of $\gamma$ are:
    \begin{table}[h!]
        \centering
        \begin{tabular}{c|c}
            Value of $\gamma$ & Type\\
            \hline
            $\gamma = 0$ & Normal Distribution or Mesokurtic\\
            $\gamma < 0$ & Flattened or Platykurtic\\
            $\gamma > 0$ & Peaked or Lepokurtic
        \end{tabular}
    \end{table}
\end{document}