\documentclass[../main.tex]{subfiles}
\begin{document}
	\chapter{Boolean Algebra}
	Let $B$ be a set of $a,b,c$ with operations sum $(+)$ and product $(\cdot)$.\newline
	Then $B$ is said to belong to the Boolean Structure if the following conditions are satisfied:
	\begin{table}[!h]
		\caption{Properties of Boolean Algebraic Structure}
		\label{boolean}
		\begin{center}
			\begin{tabular}{c|r}
				Property&Name of Property\\
				\hline
				$a+b \in B$\\$a \cdot b \in B$ & Closure Property\\
				\hline
				$a+b=b+a$\\
				$a \cdot b= b \cdot a$ & Associative Law\\
				\hline
				$a(b+c) = ab + ac $\\
				$a+bc=(a+b)(a+c)$ & Commutative Law\\
				\hline
				$\lbrace 0,1 \rbrace \in B$\\
				$a+0=a$\\
				$a+1=1$\\
				$a \cdot 0=0$\\
				$a \cdot 1=a$ & Laws of $1$ and $0$\\
				\hline
				$a+ab=a$\\
				$a(a+b)=a$ & Absorption Law\\
				\hline
				$(a+b)'=(a'b')$ & De'Morgan's Law
			\end{tabular}
		\end{center}
	\end{table}
\end{document}